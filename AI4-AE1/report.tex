\documentclass[english,11pt]{article}
\usepackage[T1]{fontenc}
\usepackage[utf8]{inputenc}
\usepackage{graphicx}
\usepackage{geometry}
\usepackage{caption}
\usepackage{subcaption}
\usepackage{babel}
\usepackage{amsmath}
\newcommand{\naive}{na\"{\i}ve\ }

\begin{document}

\title{Artificial Intelligence 4 Assessed Exercise}
\author{Motiejus Jakštys}
\date{23 November 2012}

\maketitle
\pagebreak
\tableofcontents
\pagebreak

\section{Introduction}
\subsection{Identification}
This document is the report of the Artificial Intelligence 4 Assessed
Exercise.

\subsection{Contents of the deliverable}

TODO

\section{Design}
There were 100 audio files given: 50 containing silence, and 50 containing
speech. The purpose of the exercise was to create a system which predicts the
class of a given audio file (silence or speech) using a training set. The
system was created and evaluated. Its performance is described in this
document.

\subsection{Performance}
Performance measure of the agent by the success rate of the matching
process. The more files are correctly assigned to the silence or speech
category, the better the performance.

\subsection{Environment}
Agent operates in Telephone Exchange or any environment which requires
distinction between silence and speech in a sound file.

\subsection{Actuators}
Actuators for this agent are computer screen or output file. This agent is
likely to be a part of a larger program, which would execute a more
business-oriented task, like stop a phone conversation or contribute to the
database with a statistical property of the phone conversation.

\subsection{Sensors}
Since this agent is purely software agent, its sensors are sound files.


\section{Theory}

This work consisted of two sub-tasks:
\begin{description}
    \item{Sensing:} reading the sample files, extracting energy, magnitude and
        zero crossing rate (later ZCR).

    \item{Reasoning:} applying \naive Bayes classifier to the data and
        predicting the class (speech or silence) of the given file.
\end{description}

\subsection{Sensing}

\subsubsection{Extracting Energy, Magnitude, ZCR}

At first a program was made which can extract energy, magnitude and ZCR from
the file. This is the first step in sensing: extract necessary properties from
raw data. In this work all three properties were calculated using a window of
300 samples. Here you will find energy, magnitude and ZCR described. Each
property has an associated figure with a provided sample.

Energy (fig.~\ref{fig:sample_e}) is calculated as follows:

$$ E[n] = \sum_{k=-\infty}^{\infty} s^2[k] \cdot w[n-k] $$

Where $n$ is the sample number and $w$ is a rectangular window.

Magnitude (fig.~\ref{fig:sample_m}):

$$ M[n] = \sum_{k=-\infty}^{\infty} |s[k]| \cdot w[n-k] $$

Zero crossing rate (fig.~\ref{fig:sample_z}):

$$
    Z[n] = \frac{1}{2N}
        \sum_{k=-\infty}^{\infty}
        |sign(s[k]) - sign(s[k-1])| \cdot w[n-m]
$$

\begin{figure}
    \centering
    \begin{subfigure}[b]{0.5\textwidth}
        \includegraphics[width=\textwidth]{res/sample_e.pdf}
        \caption{Energy}
        \label{fig:sample_e}
    \end{subfigure}%
    \begin{subfigure}[b]{0.5\textwidth}
        \includegraphics[width=\textwidth]{res/sample_m.pdf}
        \caption{Magnitude}
        \label{fig:sample_m}
    \end{subfigure}

    \begin{subfigure}[b]{\textwidth}
        \includegraphics[width=\textwidth]{res/sample_z.pdf}
        \caption{ZCR}
        \label{fig:sample_z}
    \end{subfigure}
    \caption{Sample audio file}\label{fig:sample}
\end{figure}

\subsubsection{Aggregating sample data}

Once it was possible to calculate energy, magnitude and ZCR, it was necessary
to aggregate the data. A value by itself does not give much information,
however, mean value of the file is a bit more valuable property. Once means of
every property for every file are known, correlation between those can be
visualised and measured.

Pearson correlation for energy--magnitude is $0.9955$
(fig.~\ref{fig:aggr_e-m}), for energy--ZCR is $-0.6041$
(fig.~\ref{fig:aggr_e-z}) and magnitude--ZCR is $-0.6281$
(fig.~\ref{fig:aggr_m-z}).

\begin{figure}
    \centering
    \begin{subfigure}[b]{0.33\textwidth}
        \includegraphics[width=\textwidth]{res/aggr_e-m.pdf}
        \caption{Energy -- Magnitude}
        \label{fig:aggr_e-m}
    \end{subfigure}%
    \begin{subfigure}[b]{0.33\textwidth}
        \includegraphics[width=\textwidth]{res/aggr_e-z.pdf}
        \caption{Energy -- ZCR}
        \label{fig:aggr_e-z}
    \end{subfigure}%
    \begin{subfigure}[b]{0.33\textwidth}
        \includegraphics[width=\textwidth]{res/aggr_m-z.pdf}
        \caption{Magnitude -- ZCR}
        \label{fig:aggr_m-z}
    \end{subfigure}
    \caption{Aggregated data}\label{fig:aggr}
\end{figure}

\subsubsection{Classification}

Our goal is to apply \naive Bayes classification to the data set, and decide
the class of the unknown file, whether it is silence or speech. We (strongly)
assume that neither of the properties that we are using in the classification
are dependant. For the first run let us close our eyes to the $0.995$ linear
correlation between energy and magnitude, run the analysis and measure the
performance. Later the most conflicting metric will be discarded, learning and
analysis will be re-ran, and performance will be compared.

Given an unknown file probability of it being silence or speech is the same:

$$ P(silence) = P(speech) = 0.5 $$

It is possible to determine probability distribution of $P(C|E)$, $P(C|M)$,
$P(C|Z)$. Our goal is to calculate $ P(silence | E, M, Z) $ from that data. In
other words, calculate probability of the file being silence, given energy,
magnitude and ZCR, given our training set. In order to get $P(cause|effect)$
(in this case cause is energy, magnitude and ZCR, and effect is silence or
speech), we have to apply Bayes and chain rules:

\begin{align}
   & P(C | E, M, Z) = \\
   & \frac{ P(E|C,M,Z) P(C|M,Z) }{ P(E|M,Z) } = \\
   & \frac{ P(E|C,M,Z) P(M|C,Z) P(C|Z) }{ P(E|M,Z) P(M|Z) } = \\
   & \frac{ P(E|C,M,Z) P(M|C,Z) P(Z|C) P(C) }{ P(E|M,Z) P(M|Z) P(Z) } =
    \label{eq:chain_rule} \\
   & \alpha P(E|C,M,Z) P(M|C,Z) P(Z|C) P(C) = \label{eq:alpha} \\
   & \alpha P(E|C) P(M|C) P(Z|C) P(C) \label{eq:independence}
\end{align}

Getting to equation~\ref{eq:chain_rule} is just applying chain rule. Since the
equation denominator is a constant (does not depend on $C$), it is equal in
both classes (silence and speech), and hence it gets a new name $\alpha$.
Getting to equation~\ref{eq:independence} is slightly more interesting. Like
mentioned before, we believe that all properties -- energy, magnitude and ZCR
-- are conditionally independent. For that reason we can simplify the mentioned
expression using \naive Bayesian classifier.

\end{document}
