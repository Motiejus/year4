\documentclass[english,11pt]{article}
\usepackage[T1]{fontenc}
\usepackage[utf8]{inputenc}
\usepackage{geometry}
\newcommand{\naive}{na\"{\i}ve\ }
\usepackage{babel}

\begin{document}

\title{Artificial Intelligence 4 Assessed Exercise}
\author{Motiejus Jakštys}
\date{23 November 2012}

\maketitle
\pagebreak
\tableofcontents
\pagebreak

\section{Introduction}
\subsection{Identification}
This document is the report of the Artificial Intelligence 4 Assessed
Exercise.

\subsection{Contents of the deliverable}

TODO

\section{Design}
There were 100 audio files given: 50 containing silence, and 50 containing
speech. The purpose of the exercise was to create a system which predicts the
class of a given audio file (silence or speech) using a training set. The
system was created and evaluated. Its performance is described in this
document.

\subsection{Performance}
Performance measure of the agent by the success rate of the matching
process. The more files are correctly assigned to the silence or speech
category, the better the performance.

\subsection{Environment}
Agent operates in Telephone Exchange or any environment which requires
distinction between silence and speech in a sound file.

\subsection{Actuators}
Actuators for this agent are computer screen or output file. This agent is
likely to be a part of a larger program, which would execute a more
business-oriented task, like stop a phone conversation or contribute to the
database with a statistical property of the phone conversation.

\subsection{Sensors}
Since this agent is purely software agent, its sensors are sound files.


\section{Theory}

This work consisted of two sub-tasks:
\begin{description}
    \item{Sensing:} reading the sample files, extracting energy, magnitude and
        zero crossing rate (later ZCR).

    \item{Reasonining:} applying \naive Bayes classifier to the data and
        predicting the class (speech or silence) of the given file.
\end{description}

\subsection{Sensing}

At first a program was made which can extract energy, magnitude and ZCR from
the file. This is the first step in sensing: extract necessary properties from
raw data. All three properties were calculated using a window of 300 samples.

Energy is calculated as follows:

$$ E[n] = \sum_{k=-\infty}^{\infty} s^2[k] \cdot w[n-k] $$

Where $n$ is the sample number and $w$ is a rectangular window.


Magnitude:

$$ M[n] = \sum_{k=-\infty}^{\infty} |s[k]| \cdot w[n-k] $$

Zero crossing rate:

$$ Z[n] = \frac{1}{2N} \
    \sum_{k=-\infty}^{\infty}
    |sign(s[k]) - sign(s[k-1])| \cdot w[n-m]
$$

\end{document}
