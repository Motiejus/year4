\documentclass[english,11pt]{article}
\usepackage[T1]{fontenc}
\usepackage[utf8]{inputenc}
\usepackage{graphicx}
\usepackage{geometry}
\usepackage{caption}
\usepackage{subcaption}
\usepackage{babel}
\usepackage{amsmath}
\usepackage[nottoc,numbib]{tocbibind}
\usepackage{verbatim}
\usepackage{hyperref}
\numberwithin{equation}{section}
\newcommand{\naive}{na\"{\i}ve\ }
\newcommand{\Naive}{Na\"{\i}ve\ }

\begin{document}

\title{Machine Learning 4 Assessed Exercise}
\author{Motiejus Jakštys}
\date{26 February 2013}

\maketitle
\pagebreak
\tableofcontents
\pagebreak

\section{Introduction}
\subsection{Identification}
This document is the report of the Machine Learning 4 Assessed Exercise.

\subsection{Contents of the deliverable}

This deliverable is a report of the assessed exercise. It includes the task
description, model, design of the application and performance evaluation.

\subsection{High-level overview of problem and solution}

\subsubsection{Problem description}

When user operates a touch-screen device, she is likely to touch inexact
locations\cite{WeiRogMur}. Errors are not random, which means that, given some
training data, the pointing accuracy can be improved.

The problem can be tackled by creating user a specific model for her touch
behaviour and applying a machine learning technique, which can infer the
intended touch location from the one reported by the device using the training
data.

For mechanism to work it needs a set of locations where the user intended to
touch alongside where the user actually touched, reported by the device. That
can be achieved by a game or some other more or less intrusive way, which is out
of the scope of this document.

This report describes a machine-learning approach which solves the problem. In
highest level it learns user behaviour by having some training touch data (300
touches per user), and then, given a new set of touch coordinates, returns an
inferred touch location.

The approach adopted in this work is Gaussian Process, which is described in
detail in \cite{WeiRogMur}.

\clearpage

\begin{thebibliography}{9}
    \bibitem{WeiRogMur}

    Daryl Weir, Simon Rogers, Markus L\"ochtefeld, and Roderick Murray-Smith. A
    user-specific machine learning approach for improving touch accuracy on
    mobile devices. \emph{In Proceedings of the 25th ACM Symposium on User
    Interface Software and Technology}, 2012.

\end{thebibliography}
\end{document}
