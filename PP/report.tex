\documentclass[english,11pt]{article}
\usepackage[T1]{fontenc}
\usepackage[utf8]{inputenc}
\usepackage{graphicx}
\usepackage{geometry}
\usepackage{caption}
\usepackage{subcaption}
\usepackage{babel}
\usepackage{amsmath}
\usepackage[nottoc,numbib]{tocbibind}
\usepackage{verbatim}
\usepackage{hyperref}
\numberwithin{equation}{section}
\newcommand{\naive}{na\"{\i}ve\ }
\newcommand{\Naive}{Na\"{\i}ve\ }
\newcommand{\listing}[1]{
    \subsubsection{#1}
    \label{lst:#1}
    {\small \verbatiminput{#1}}
}

\begin{document}

\title{Personal Project Report\\
    Fast Genuine Generalized Consensus (FGGC)}
\author{Motiejus Jakštys}
\date{1 January 2013}

\maketitle
\pagebreak
\tableofcontents
\pagebreak

\section{Introduction}
\subsection{Identification}

This document is the report of Year 4 Personal Project named "Implement and
test the Fast Genuine Generalized Consensus algorithm for Networks-on-Chip".

\section{Outline}

What is FGGC?

FGGC belongs to the family of Paxos algorithms. Its characteristics are similar
to Fast Paxos, but with better recovery options.

From abstract[1]:

Consensus is a central primitive for building replicated systems, but its
latency constitutes a bottleneck. A well-known solution to consensus is Fast
Paxos. In a recent paper, Lamport enhances Fast Paxos by leveraging the
commutativity of concurrent commands. The new primitive, called Generalized
Paxos, reduces the collision rate, and thus the latency of Fast Paxos. However
if a collision occurs, the latency of Generalized Paxos equals six
communication steps, which is higher than Fast Paxos. This paper presents FGGC,
a novel consensus algorithm that reduces recovery delay when a collision occurs
to one.

\subsection{Motivation}

During the last decade Moore's law had changed the rules of machine
performance. Clock speeds used to double every three years [reference needed].
However, recently frequency limits have been reached due to minimum possible
sizes of the chips. On the other hand, given clock speed limits, number of
cores on a chip started increasing rapidly. 8-core CPUs are now very common in
commodity desktop systems, and servers with 24 cores are more and more widely
deployed [reference needed].

Downside of current approach (SMP or NUMA) is that memory is shared among the
cores. Every core has its own memory cache, which must be consistent with
caches of other cores. This requirement increases communication overhead
between cores solely for cache coherency exponentially as number of cores
increases. Therefore for many cores (hundreds or thousands) another memory
management approach is needed.

Tilera is a multi-core machine producer which takes novel other approach in
memory management. Main (DDR3) memory is shared amongst the cores, however,
what happens with caches is left completely up to the programmer to decide.
For example, cache coherency can be turned off completely. A send very low
latency and high bandwidth messages to other cores or their caches, so
the need to have consistent memory on all the cores decreases.

With this approach it is more natural to look at every core like to an
independent worker, or actor, and think about data exchange not using shared
memory, but sending messages.

It is estimated that within 5 years a chip with 65536 cores will be produced
[reference needed].
