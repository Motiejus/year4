\documentclass[english,11pt]{article}
\usepackage[T1]{fontenc}
\usepackage[utf8]{inputenc}
\usepackage{graphicx}
\usepackage{amsmath}
\usepackage{geometry}
\usepackage{caption}
\usepackage{subcaption}
\usepackage{babel}
\usepackage[nottoc,numbib]{tocbibind}
\usepackage{hyperref}
\numberwithin{equation}{section}

\begin{document}

\title{Bioinformatics Writing Exercise 2}
\author{Motiejus Jakštys}
\date{28 February 2013}

\maketitle
\pagebreak
\tableofcontents
\pagebreak

\section{Introduction}

In this text we overview Next Generation Sequencing Technology. In the beginning
we introduce what it does and what it is for, briefly how it works and how
reliable is it. Followed by current users of the technology and reasons for
using it. In the end my opinion on ethical implications are given.

\section{Introduction to sequencing}

\subsection{What is sequencing}

Genome sequencing is a process of determining organism DNA. In other words,
given a tissue, goal is to determine the series of ATCG's that form the organism
DNA.

There are two approaches to sequence DNA: \emph{de novo} and \emph{shotgun}.
sequencing. I will briefly describe both.

\subsection{Shotgun method sequencing}

Shotgun method sequencing means breaking the DNA strand to many small pieces and
aligning them to a reference sequence. This method is cheap and fast, however,
it requires a reference DNA \emph{a priori} in order for it to be decent
quality.

Sometimes, if reference DNA for the target species does not exist, a similar
organism can be taken as a reference.

On the other hand, there are ways to align the DNA using shotgun sequencing
without a reference genome. However, it is significantly lower
quality~\cite{shotgun}.

Shotgun method allows to make really short reads (75 bp) and get decent quality
results. Shorter reads means less restrictions to technology and therefore
cheaper sequencing.

\subsection{\emph{De novo} sequencing}

\emph{De novo} sequencing is when reference DNA does not exist, and DNA must be
aligned with itself. Because there is nothing to align to, longer reads have
significant advantage which reasons are out of the scope of this report.

\section{Technology}

Sequencing roughly works as follows. DNA is split into many small pieces which
are then algorithmically matched together during post-processing stage. Pieces
of DNA are assembled together by matching overlapping regions. Which means with
longer reads comes higher quality DNA matches.

Most next generation sequencers, as we can see later, produce relatively short
sequences, and therefore are used for shotgun method.

\subsection{Sanger method}

Though Sanger method is not "next generation sequencing method", but we will
introduce it for completeness because of its historical
importance~\cite{sanger}. Sanger method was quite significant in human genome
project initial genome assembly. Reads are quite long (averaging 750 bp, up to
1500 bp). Comparing to other methods (described below), this method yields one
of the longest reads. Sanger method initially was heavily used in human genome
project for \emph{de novo} assembly. It costs roughly \$2400-5000 to sequence
one million base pairs~\cite{ng-comparison}

\subsection{Next generation sequencing}

Sanger method was called "first generation" sequencing method, after which "next
generation" sequencing methods emerged. Their main properties are ultra high
throughput and massive parallelism. Notable examples are Pacific Bio (\$2/Mb
with 2900 mb reads, very expensive equipment), Ion Torrent (\$1/Mb, 200 reads,
cheaper equipment), Illumina (\$0.05 to \$0.15 per Mb, 50-250 bp
reads)~\cite{ng-comparison2}.

The method for sequencing highly depends on the application: price, time, base
pair length and other factors should be taken into account when choosing the
sequencing method and equipment.

\section{Uses of DNA sequencing}

Users of the sequencing technology are logically organizations that need DNA
sequences. Most of the users are "biotechnologists", but this label is
diverse~\cite{biotech}.

Biotechnology is science which is concerned by modifying organisms in order to
satisfy various human needs. Notable examples are medicine, agriculture,
bioremediation and education.

Pharmacogenomics is the science which uses DNA to make up his/her response to
drugs~\cite{pharmacogenomics} and develop new ones/improve existing ones using
that information.

For agricultural needs DNA is used in order to improve existing crops or animals
by modifying their DNA in a predicted manner. For instance, to reduce
vulnerability to harsh environments for certain kinds of crops~\cite{drought}.

Bioremedation is used to adapt or create microorganisms that can be used to
clean up contaminated environments. One good example is hydrocarbonoclastic
bacteria (HCCB), which is used for oil degradation in cases of accidental
gasoline spillage~\cite{oil-degradation}.

DNA sequences are also used by biologists and antropologists in order to trace
evolution. Aligning DNAs of different species can tell a lot how the species
evolved.

\section{Ethics}

Since biotechnology is all about modifying living organisms, it is seen as a
very intrusive human action to the "laws of nature". Topic of direct genome
manipulation of organisms is highly contraversal~\cite{food}, especially for
modifying plants for food purposes.

In the future good knowledge of DNA and its mechanisms will arguably techically
allow to make predictable changes in our offspring. This is going to be very
controversial when this happens, too.

In my opinion, I am confident by doing anything with any genome, as long as its
primary purpose is not self destruction. I am perfectly happy with plant,
animal, human genome modifications, if it makes lives for everyone easier. This
includes crop and animal improvements, human genome modifications for resistance
to diseases and etc.

On the other hand, I would not want my offspring to be pre-defined before
his/her birth. Unpredictable life and less safe, in a sense, is way more
interesting.

\begin{thebibliography}{9}

    \bibitem{shotgun}
        J. Craig Venter, Mark D. Adams, Granger G. Sutton, Anthony R. Kerlavage,
        Hamilton O. Smith, Michael Hunkapiller. \emph{Science}, 5 June 1998:
        Vol.  280 no. 5369 pp. 1540-1542.

    \bibitem{sanger}
        Sanger F, Nicklen S, Coulson AR. DNA sequencing with chain-terminating
        inhibitors. \emph{Proc Natl Acad Sci USA}. 1977 Dec; 74(12):5463-7.

    \bibitem{ng-comparison}
        Lin Liu, Yinhu Li, Siliang Li, Ni Hu, Yimin He, Ray Pong, Danni Lin,
        Lihua Lu, and Maggie Law.  Comparison of Next-Generation Sequencing
        Systems \emph{J Biomed Biotechnol}. 2012; 2012: 251364.

    \bibitem{ng-comparison2}
        Quail, Michael; Smith, Miriam E; Coupland, Paul; Otto, Thomas D; Harris,
        Simon R; Connor, Thomas R; Bertoni, Anna; Swerdlow, Harold P; Gu, Yong.
        "A tale of three next generation sequencing platforms: comparison of Ion
        torrent, pacific biosciences and illumina MiSeq sequencers". \emph{BMC
        Genomics} 13 (1): 341.

    \bibitem{biotech}
        Thieman, W.J.; Palladino, M.A. (2008). \emph{Introduction to
        Biotechnology}. Pearson/Benjamin Cummings. 

    \bibitem{pharmacogenomics}
        Ermak G., Modern Science and Future Medicine (second edition), 164 p.,
        2013

    \bibitem{drought}
        Sara Abdulla (27 May 1999). "Drought stress". \emph{Nature News.}
        doi:10.1038/news990527-9.

    \bibitem{oil-degradation}
        "Genomic Insights into Oil Biodegradation in Marine Systems".
        \emph{Microbial Biodegradation: Genomics and Molecular Biology.} Caister
        Academic Press.

    \bibitem{food}
        L. Frewera, J. Lassenb, B. Kettlitzc, J. Scholdererd, V. Beekmane, K.G.
        Berdalf Societal aspects of genetically modified foods. \emph{Food and
        Chemical Toxicology}, Volume 42, Issue 7, July 2004, p. 1181–1193.

\end{thebibliography}

\end{document}
